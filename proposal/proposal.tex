















% DO NOT EDIT ANYMORE
% This version has been sent :)















\documentclass{ncc}
\usepackage{multicol}
\usepackage[ampersand]{easylist}
%\usepackage[top=1cm, bottom=2cm, left=2.5cm, right=2.5cm]{geometry}
\usepackage[top=3cm, bottom=3cm, left=4.5cm, right=4.5cm]{geometry}
\usepackage[hidelinks]{hyperref}
\pagenumbering{gobble}


\begin{document}

\title{SSN Project Proposal: \\ 
Practical Passphrase Cracking}
\author{Dirk Gaastra \textperiodcentered{} Lennart van Gijtenbeek \textperiodcentered{} Luc Gommans}
\maketitle

\begin{center}
{\large September 21st, 2017}
\end{center}


\section{Introduction}

For as long as users have shared computer systems, people have implemented
authentication methods. Most commonly, this was in the form of a username and
password: the username identifies a user and the password authenticates. Since
computers are becoming ever faster, people need to choose ever stronger
passwords. Where shorter passwords provided sufficient protection in the past,
it is now recommended to use completely random strings of at least twelve
characters. However, memorizing twelve or more random characters is not reasonable for most people. 

A passphrase is proposed as an alternative that is easier to remember. They are
short strings, consisting of a few random (dictionary) words. Only a few words
can already be as strong as a strong password. Most of the research into the topic has been on passwords rather than passphrases, and most of the cracking tools only work well for passwords. In this work, we will attempt practical attacks on passphrases, aiming to further the research in this field. The approach we take in this investigation, is to reduce the passphrase search space by incorporating personal data about the target, in order to generate more likely passphrases (using a self-implemented cracking tool).


\section{Research Questions}

%Perhaps we should make this into more of a story, instead of bullet points..? OR do both bullet points in the end, starting with a story.

We have outlined several research questions in this section. The investigation
consists of both theoretical and practical aspects. Please see these question
below:

\begin{enumerate}
\item To what degree [quantify] do users incorporate interests in their passphrases?
\item How feasible is it for an attacker to crack a passphrase when a victim's interests are known?
\item Can we significantly reduce the passphrase search space using an automated tool?
\item What resources are required to {generate such a database of interests (storage) / brute a passphrase given a database (CPU)}?
\end{enumerate}


\section{Scope}

Previous research has been done into predictability of passphrases (see section
\ref{sec:relatedwork}) and tools have been built which can be used to crack
passwords. To the best of our knowledge, there exists no practical passphrase
cracking tool which can be seeded with user's interests and/or hobbies. For
instance, if we know someone is into a specific band, we can focus our search
efforts by generating passphrases that incorporate their lyrics, and thereby
effectively decrease the search space. This transforms the problem of cracking
a passphrase from an impossible task (with respect to computational time) to
more manageable proportions. The focus of this research will therefore be on
ways to seed a passphrase cracking tool with useful information. This tool will
then work out passphrases that incorporate the information that was used to
seed the tool.

It is out of scope to determine the interests of users in an automated fashion.
We assume that the interests of a person are given. By means of a survey we
will figure out which themes are prevalent in users' passphrases. We will not
attempt to create multiple, comprehensive datasets for various topics; instead,
we will attempt to construct such a database for a specific prevalent topic. 


\section{Implementation}

The implementation of the tool will probably be done in {\it Java}. This
language is fast enough for our purposes, while at the same time providing the
functionality we need. The seed database will have a custom format, that will be loaded into memory by the tool.

\section{Planning}

\textbf{Week 1}

\begin{itemize}
	\item Perform a survey to collect passphrases and personal data.
	\item Set up a seed database, for 1 or 2 specific hobbies.
	\item Determine the design/coding plan for the cracking tool.
	\subitem -- Review literature for potential good designs.
\end{itemize}

\noindent \textbf{Week 2}
\begin{itemize}
	\item Start building the tool.
	\item At the end of this week, the tool should have basic functionality.
\end{itemize}

\noindent \textbf{Week 3}
\begin{itemize}
	\item Continue improving the tool on performance and accuracy.
	\item Start writing on the project report (draft version) at the end of this week.
\end{itemize}

\noindent \textbf{Week 4}
\begin{itemize}
	\item The tool should be finished at the start of this week.
	\item Extensively test the tool, using the survey data for validation purposes.
	\item Reflect on the research questions.
	\item Finalize the project report.
	\item Prepare presentation.
\end{itemize}


\section{Ethical Considerations}

For this research we will do a survey, asking people to come up with a couple
of different passphrases and list things such as their hobbies, favorite bands,
and favorite TV shows. For improved results, this survey will be anonymous,
such that participants are encouraged to fill in the survey truthfully. Based
on the data obtained through the survey, one might be able to identify
individuals; however, the chance of this happening is unlikely and the impact
negligible (since the passphrases are supposed to be fictitious).

We would like to see our final product (the passphrase cracker) in action in
order to test it. This means that we might have to find a dataset containing
passwords linked to real user accounts. Publishing the code to any tool used
for cracking can have initial negative effects on the overall security of
systems. However, our aim is that this will only increase the security
practices of people and systems in the long run.


\section{Related work}\label{sec:relatedwork}

The benefits of using passphrases are discussed by \cite{keith2009behavioral}
and \cite{yan2000memorability}; these papers claim that with good design, the
security of passwords increases dramatically while not increasing the amount of
login failures.

In terms of research into cracking, P. Sparell et al. have looked into using
Markov chains to predict the next word most often used when given a few words
\cite{sparell2016linguistic}. This research can provide us with some insight
into how we can construct our passphrases.

Furthermore, A. Roa et al. have gotten interesting results when accounting for
grammatical structures while cracking passwords \cite{rao2013effect}. Most
interestingly, they showed that using longer passphrases did not mean they were
more secure. They also managed to decrease the search space for passphrases by
about 50\% by being grammatically aware.

While not specific for passphrases (even though they are included),
\cite{weir2010using} gives insight into making 'educated' password guesses
instead of just brute-forcing it. The paper is very extensive and will be
helpful to get a better idea of how to predict passphrases.

For a more hands-on approach, \cite{labrandecrack} demonstrated that using
publicly-available data, one can build a large dictionary with famous sentences
which can be used to crack millions of passphrases, reaffirming previous
research stating that a longer password does not equal more security.
% e.g. try brainyquote things as passphrases


\bibliographystyle{unsrt}
\bibliography{bibliography}

\end{document}

